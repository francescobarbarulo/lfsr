\documentclass[11pt,a4paper,oneside, openright]{article}
\usepackage{graphicx}
\usepackage{subfig}
\usepackage[british]{babel}
\usepackage[utf8]{inputenc}
\usepackage{mathtools}
\usepackage{setspace}
\usepackage{verbatim}
\usepackage{listings}
\usepackage{color}
\usepackage[hyphens]{url}
\usepackage[bookmarks]{hyperref}

\definecolor{gray}{rgb}{0.95,0.95,0.95}
\lstset{
    backgroundcolor=\color{gray},
    basicstyle=\footnotesize\ttfamily,
    numbers=left,
    tabsize=1
}

% \renewcommand{\baselinestretch}{1.5}

\begin{document}
{\setstretch{1.0}
  \begin{titlepage}
  	\centering
  	\includegraphics[width=6cm]{images/unipi.eps}\par
  	\vspace{1cm}
    {\Large Department of Information Engineering \par}
  	\vspace{2cm}
  	{\huge\textsc{Linear Feedback Shift Register}\par}
    \vspace{0.5cm}
  	{\Large Electronics and Communication Systems\par}
  	\vspace{2cm}
  	{\Large Francesco \textsc{Barbarulo}\par}

  	\vfill

    % Bottom of the page
  	{\large 2018-2019\par}
  \end{titlepage}
}


\tableofcontents

\newpage

\section{Introduction}
A Linear-Feedback Shift Register (LFSR) is a shift register whose input bit is a linear function of its previous state.

The most commonly used linear function of single bits is exclusive-or (XOR). Thus, an LFSR is often a shift register, whose input bit is driven by the XOR of some bits of the overall shift register value.

The LFSR can be initialized with an initial value called seed and, as the operation of the register is deterministic, the stream of values produced by the register is completely determined by its current state. Likewise, as the register has a finite number of possible states, it can enter a repeating cycle. However, a LFSR with a well-chosen feedback function can produce a sequence of bits that appears random and has a very long cycle.

It is necessary to ensure that the LFSR never enters an all-zeros state, which is illegal, because the register would remain locked-up in this state.

A LFSR is described by its feedback polynomial. The polynomial needs a number of bits according to its degree. Furthermore, it has different periods of the repeating cycle: higher the degree is, longer the period is.

There are two architecture configurations through which a LFSR can be implemented, one is called Fibonacci LFSR and the other is called Galois LFSR \cite{lfsr}.

\begin{figure}[!b]
    \centering
    \includegraphics[width=\textwidth]{images/lfsr-f16}
    \caption{A 16-bit Fibonacci LFSR}
    \label{fig:lfsr-f16}
\end{figure}

\subsection{Fibonacci LFSR}
The bit positions that affect the next state are called taps. The rightmost bit of the LFSR is the output bit. The taps are sequentially XORed with the output bit and then fed back into the leftmost bit. The sequence of bits in the rightmost position is the output stream.

The arrangement of taps for feedback in a LFSR can be expressed as a polynomial mod 2. This means that the coefficients of the polynomial must be 1s or 0s. The powers of the terms of the feedback polynomial represent the tapped bits, counting from the left.

In some conditions, the LFSR can cycle through all possible states; in this case, the LFSR is maximal-length. This happens if and only if the corresponding feedback polynomial is primitive. Thus, the necessary conditions are:
\begin{itemize}
    \item the number of tap is even;
    \item the set of taps is setwise co-prime (there must be no divisor, apart from 1, common to all taps).
\end{itemize}

Below, there is a C code example for the 16-bit maximal-period Fibonacci LFSR represented in Figure~\ref{fig:lfsr-f16}:

\lstinputlisting[language=C, linerange={4-13,16-22}, caption={Fibonacci LFSR C program}, label={lst:fibonacci-c}, captionpos=b]{../code/lfsr.c}


\subsection{Galois LFSR}
Named after the French mathematician Évariste Galois, a LFSR in Galois configuration, which is also known as modular, internal XORs, or one-to-many LFSR, is an alternative structure that can generate the same output stream as a conventional LFSR (but offset in time). In the Galois configuration, when the system is clocked, bits that are not taps are shifted one position to the right unchanged. The taps, on the other hand, are XORed with the output bit before they are stored in the next position. The new output bit is the next input bit. The effect of this is that, when the output bit is zero, all the bits in the register shift to the right unchanged, and the input bit becomes zero. When the output bit is one, the bits in the tap positions all flip (if they are 0, they become 1, and if they are 1, they become 0), and then the entire register is shifted to the right and the input bit becomes 1.

To generate the same output stream, the order of the taps must be the counterpart of the order for the conventional LFSR, otherwise the stream will be in reverse. The internal state of the LFSR is not necessarily the same. A time offset exists between the streams, so a different startpoint will be needed to get the same output each cycle.

Galois LFSRs do not concatenate every tap to produce the new input (the XORing is done within the LFSR, and no XOR gates are run in serial; therefore, the propagation times are influenced only by one XOR, rather than by a whole chain), thus it is possible for each tap to be computed in parallel, increasing the speed of execution.

Below there is a C code example of the 16-bit maximal-period Galois LFSR represented in Figure~\ref{fig:lfsr-g16}:

\lstinputlisting[language=C, linerange={4-10,12-23}, caption={Galois LFSR C program}, label={lst:galois-c}, captionpos=b]{../code/lfsr-g.c}

\begin{figure}
    \centering
    \includegraphics[width=\textwidth]{images/lfsr-g16}
    \caption{A 16-bit Galois LFSR}
    \label{fig:lfsr-g16}
\end{figure}

\newpage

\section{Applications}
LFSRs can be implemented in hardware, and this makes them useful in applications \cite{lfsr} that require a very fast generation of a pseudo-random sequence.

\subsection{Uses as counters}
The repeating sequence of states of a LFSR allows it to be used as a clock divider or as a counter when a non-binary sequence is acceptable, as it is often the case where the computer index or the framing locations need to be machine-readable. LFSR counters have simpler feedback logic than natural binary counters or Gray-code counters and, therefore, they can operate at higher clock rates. One can obtain any other period by adding to a LFSR, which has a longer period, some logic that shortens the sequence by skipping some states.

\subsection{Uses in cryptography}
LFSRs have long been used as pseudo-random number generators in stream ciphers (especially in military cryptography), thanks to the ease of construction from simple electromechanical or electronic circuits, to long periods and very uniformly distributed output streams. However, a LFSR is a linear system, which leads to easy cryptanalysis.

Three general methods are employed to reduce this problem in LFSR-based stream ciphers:
\begin{itemize}
    \item Non-linear combination of several bits from the LFSR state;
    \item Non-linear combination of the output bits of two or more LFSRs;
    \item Irregular clocking of the LFSR, as in the alternating step generator.
\end{itemize}

Important LFSR-based stream ciphers include A5/1 and A5/2, used in GSM cell phones, E0, used in Bluetooth, and the shrinking generator. The A5/2 cipher has been broken and both A5/1 and E0 have serious weaknesses.

\subsection{Uses in circuit testing}
LFSRs are used in circuit testing for test-pattern generation and for signature analysis.

\paragraph{Test-pattern generation}
Complete LFSR are commonly used as pattern generators for exhaustive testing, since they cover all possible inputs for an n-input circuit. Maximal-length LFSRs and weighted LFSRs are widely used as pseudo-random test-pattern generators for pseudo-random test applications.

\paragraph{Signature analysis}
In built-in self-test (BIST) techniques, storing all the circuit outputs on chip is not possible, but the circuit output can be compressed to form a signature that will later be compared to the golden signature (of the good circuit) to detect faults. Since this compression is lossy, there is always a possibility that a faulty output also generates the same signature as the golden signature and the faults cannot be detected. This condition is called error masking or aliasing. BIST is accomplished with a multiple-input signature register (MISR or MSR), which is a type of LFSR. A standard LFSR has a single XOR or XNOR gate, where the input of the gate is connected to several "taps" and the output is connected to the input of the first flip-flop. A MISR has the same structure, but the input to every flip-flop is fed through an XOR/XNOR gate. For example, a 4-bit MISR has a 4-bit parallel output and a 4-bit parallel input. The input of the first flip-flop is XOR/XNORed with parallel input bit zero and the "taps". Every other flip-flop input is XOR/XNORd with the preceding flip-flop output and the corresponding parallel input bit. Consequently, the next state of the MISR depends on the last several states opposed to just the current state. Therefore, a MISR will always generate the same golden signature, given that the input sequence is the same every time.

\subsection{Uses in digital broadcasting and communications}
To prevent short repeating sequences (e.g., runs of 0s or 1s) from forming spectral lines that may complicate symbol tracking at the receiver or interfere with other transmissions, the data bit sequence is combined with the output of a linear-feedback register before modulation and transmission. This scrambling is removed at the receiver after demodulation. When the LFSR runs at the same bit rate as the transmitted symbol stream, this technique is called scrambling. When the LFSR runs considerably faster than the symbol stream, the LFSR-generated bit sequence is called chipping code. The chipping code is combined with the data using exclusive or before transmitting using binary phase-shift keying or a similar modulation method. The resulting signal has a higher bandwidth than the data and, therefore, this is a method of spread-spectrum communication. When used only for the spread-spectrum property, this technique is called direct-sequence spread spectrum; when used to distinguish several signals transmitted in the same channel at the same time and frequency, it is called code division multiple access.

\section{Architecture}

The assignment of this project is the realization of a LFSR with the following feedback polynomial:
$$ 1 + x^{11} + x^{13} + x^{14} + x^{16} $$

According to the degree of feedback polynomial, the register needs to store 16 bits where the taps are at the 16th, 14th, 13th and 11th bits.

The LFSR has to be able to be initilized, hence the \texttt{lfsr\_i} must be on 16 bits. Also the \texttt{lfsr\_o}, which represents the output value of the register, must be on 16 bits. Furthermore, there are one pin for the active low \texttt{rst\_n} in order to reset the register, and one pin for the \texttt{clk}. In order to feed the register with the initial state there is one more pin \texttt{en}.

The entity of the register is represented in Figure~\ref{fig:register}.

\begin{figure}[h]
    \centering
    \includegraphics[width=0.9\textwidth]{images/register}
    \caption{LFSR entity}
    \label{fig:register}
\end{figure}

\subsection{Use of D-flip-flop with enable}

\begin{figure}[h]
    \centering
    \includegraphics[width=0.9\textwidth]{images/dff-en}
    \caption{D-flip-flop with enable}
    \label{fig:dff-en}
\end{figure}

Internally, the LFSR is composed by 16 D-flip-flop with enable. Each of them is composed by 5 inputs and 1 output:
\begin{itemize}
    \item input \texttt{a}, which takes the i-th bit of the seed;
    \item input \texttt{d}, which takes, at the first register that stores the most significant bit, the XORed value of the taps, whereas, at all the other registers, the output of the previous register;
    \item input \texttt{en}, on which depends the current value of \texttt{d\_s}: if \texttt{en} is equal to '0', \texttt{d\_s} takes the value of \texttt{a}, else it takes the value of \texttt{d};
    \item input \texttt{clk}, which is the clock of the D-flip-flop;
    \item input \texttt{rst\_n}, which is the active low reset for the D-flip-flop;
    \item output \texttt{q}, which is the output of the register.
    \end{itemize}
  
Below is the VHDL of the \texttt{dff\_en} entity according to Figure~\ref{fig:dff-en}:

\lstinputlisting[language=VHDL, linerange={4-39}, caption={Dff\_en VHDL}, label={lst:dff-en-vhdl}, captionpos=b]{../dff_en.vhd}

\subsection{LFSR entity}
As it is explained in Figure~\ref{fig:lfsr}, for the overall implementation of the LFSR has been chosen the Fibonacci architecture, which is the conventional one. The \texttt{dff\_en} are placed in serial, however they can be loaded in parallel assigning \texttt{lfsr\_i(i)} to \texttt{a} input of the i-th \texttt{dff\_en}.

\begin{figure}[h]
    \centering
    \includegraphics[width=\textwidth]{images/lfsr}
    \caption{LFSR architecture}
    \label{fig:lfsr}
\end{figure}

The entity \texttt{lfsr} is defined as following:

\lstinputlisting[language=VHDL, linerange={4-18}, caption={LFSR entity}, label={lst:lfsr-entity-vhdl}, captionpos=b]{../lfsr.vhd}

The component of the architecture represented by the \texttt{dff\_en} is declared as following:

\lstinputlisting[language=VHDL, linerange={22-37}, caption={LFSR component}, label={lst:lfsr-component-vhdl}, captionpos=b]{../lfsr.vhd}

A signal \texttt{d\_i} is used to store the current state of the entire register. Note that it is on 17 bits:

\lstinputlisting[language=VHDL, linerange={41-41}, caption={LFSR signal}, label={lst:lfsr-signal-vhdl}, captionpos=b]{../lfsr.vhd}

The generation of the 16 d-flip-flops and their port map are the following:

\lstinputlisting[language=VHDL, linerange={45-54}, caption={LFSR components generation}, label={lst:lfsr-generation-vhdl}, captionpos=b]{../lfsr.vhd}

Finally, the output and the 17th most significant bit of \texttt{d\_i}, which represents the feedback input, are mapped as following:

\lstinputlisting[language=VHDL, linerange={57-57, 60-60}, caption={LFSR output and feedback input mapping}, label={lst:lfsr-mapping-vhdl}, captionpos=b]{../lfsr.vhd} 

\newpage

\section{Test plan}
The test plan is made using a bottom-up approach which foresees to test less complex components up to more complex architecture. For the LFSR, the objectives of tests are:
\begin{itemize}
    \item test the correctness of the \texttt{dff\_en} on which the LFSR is based;
    \item test the correctness of the LFSR.
\end{itemize}

The correctness is verified using known inputs and comparing the outputs with the expected ones. In this case, since the LFSR outputs are deterministic, it is quite simple.
 
Tests are done through testbenches that allow to simulate the behaviour of the components specifying the clock and the input values.

Simulations are made with Modelsim software.

\subsection{Dff\_en test}
The \texttt{dff\_en} is tested in the following way: at first, \texttt{en} is setted to '0' in order to see whether \texttt{q} is getting the value of \texttt{a}; then, \texttt{en} is setted to '1' and \texttt{q} should get the values of \texttt{d}.

Below, there is the testbench for the \texttt{dff\_en}:

\lstinputlisting[language=VHDL, caption={Dff\_en testbench}, label={lst:dff-en-tb}, captionpos=b]{../dff_en_tb.vhd}

The simulation output is shown in Figure~\ref{fig:dff-en-sim}.

\begin{figure}[h]
    \centering
    \includegraphics[width=0.9\textwidth]{images/dff-en-sim}
    \caption{Dff\_en simulation}
    \label{fig:dff-en-sim}
\end{figure}

\subsection{LFSR test}
The LFSR is tested feeding the input with \texttt{0xACE1} value and comparing the output with the one got from the run of the code shown in Listing~\ref{lst:fibonacci-c}.

Below, there is the testbench for the LFSR:

\lstinputlisting[language=VHDL, caption={LFSR testbench}, label={lst:lfsr-tb}, captionpos=b]{../lfsr_tb.vhd}

The simulation output is shown in Figure~\ref{fig:lfsr-sim}.

\begin{figure}[h]
    \centering
    \includegraphics[width=0.9\textwidth]{images/lfsr-sim}
    \caption{LFSR simulation}
    \label{fig:lfsr-sim}
\end{figure}

\newpage

\section{Logic Synthesis}
The logic synthesis has been done on Xilinx Vivado Design Suite in an automatic way. We have created a new project adding all the VHDL code sources, choosing the xc7z010clg400-1 device. Then, we runned the RTL analysis. The output of the elaborated design is shown in Figure~\ref{fig:elaborated-design}.

\begin{figure}[h]
    \centering
    \includegraphics[width=\textwidth]{images/elaborated-design}
    \caption{Elaborated design}
    \label{fig:elaborated-design}
\end{figure}

Then, we have launched the synthesis with clock constraint equal to 8ns (125MHz). At the end of the synthesis the following warning came up: \texttt{[Constraints 18-5210] No constraint will be written out.} According to the Xilinx Forum\footnote{\url{https://forums.xilinx.com/t5/Synthesis/Meaning-of-synthesis-warning-Constraints-18-5210-No-constraint/m-p/881007}} this warning can be safely ignored.

\subsection{Maximum clock frequency and critical path}
Timing results are useful to check whether the $ T_{sup} $ and $ T_{hold} $ rules are respected:

$$ \begin{cases} T_{clk} \geq T_{su} + T_{p-logic} + T_{c-q} \\ T_{hold} \leq T_{cd-logic} + T_{cd-q} \end{cases} $$

Vivado offers the possibility to see the timing summary of the synthesis, which is shown in Table~\ref{tab:timing}, in order to check the above conditions.

\begin{table}[h]
    \centering
    \begin{tabular}{c c c}
        WNS & WHS & WPWS \\
        6,115ns & 0,288ns & 3,500ns
    \end{tabular}
    \caption{Design timing summary}
    \label{tab:timing}
\end{table}

Hence, the clock constraint of 8ns is met. Futhermore, the Worst Negative Slack ($T_{slack}$) is quite huge and it can be decreased, decrementing the clock period.

In fact, $T_{slack}$ represents the amount of time in which the \textit{input} is stable before the clock edge, in addition to the setup time $T_{su}$. The minimum clock period $T_{clk_{min}}$ can be found as following:

$$ T_{clk_{min}} = T_{clk} - T_{slack} $$

With $ T_{clk} = 8ns $ and $ T_{slack} = 6.115ns $, 

$$ T_{clk_{min}} = 8 - 6,115 = 1,885ns $$

A synthesis with clock constraint equal to 1,885ns confirmed the above calculations returning a null slack time, but it also reported a Worst Pulse With Slack (WPWS) equal to -0,270. This means that there is a violation on the \textit{min period} of the clock which, instead, has to be at least equal to 2,155ns.

After the final synthesis with a clock constraint equal to 2,155ns, all constraints are met and the maximum clock frequency is:

$$ f_{max} = \frac{1}{T_{clk_{min}}} = 464,037 \text{MHz} $$

The resulting worst critical path is shown in Figure~\ref{fig:critical-path-synth}.

\begin{figure}[h]
    \centering
    \includegraphics[width=\textwidth]{images/critical-path-synth}
    \caption{Critical path after logic synthesis}
    \label{fig:critical-path-synth}
\end{figure}

The path with the worst slack time goes from \textit{dff\_en\_8} to \textit{dff\_en\_5}.

\newpage

\section{Implementation}
The implementation has been runned in order to know which Zybo resources would have been used if the circuit had been loaded to the board.
The implementation timing summary is different from that of the synthesis, because the latter reports \textit{estimated} times without taking in account placement and routing information. In fact, after the implementation, the new $T_{slack}$ is equal to 0,433ns and the new critical path is shown in Figure~\ref{fig:critical-path-impl}.

\begin{figure}[h]
    \centering
    \includegraphics[width=\textwidth]{images/critical-path-impl}
    \caption{Critical path after implementation}
    \label{fig:critical-path-impl}
\end{figure}

The path with the worst slack time goes from \textit{dff\_en\_2} to \textit{dff\_en\_1}.

Some warnings are thrown:
\begin{itemize}
    \item \texttt{[Drc 23-20] Rule violation (NSTD-1) Unspecified I/O Standard - 35 out of 35 logical ports use I/O standard (IOSTANDARD) value 'DEFAULT', instead of a user assigned specific value. [...]}\footnote{\url{https://www.xilinx.com/support/answers/56354.html}}
    \item \texttt{[Drc 23-20] Rule violation (UCIO-1) Unconstrained Logical Port - 35 out of 35 logical ports have no user assigned specific location constraint (LOC). [...]}\footnote{\url{https://www.xilinx.com/support/answers/59742.html}}
    \item \texttt{[ZPS7-1] The PS7 cell must be used in this Zynq design in order to enable correct default configuration.}\footnote{\url{https://forums.xilinx.com/t5/Welcome-Join/How-to-instantiate-the-PS7-block/m-p/333953}}
\end{itemize}

The first two warnings notify that I/O pins have not been assigned to the physical ones. The last one refers to the ARM Processor environment, which is out of our scopes.

\subsection{Resources}
In Figure~\ref{fig:resources} are shown the hardware resources needed for LFSR implementation on the Zybo:

\begin{itemize}
    \item 9 x slice;
    \item 17 x LUT;
    \item 32 x FDCE.
\end{itemize}

\begin{figure}[h]
    \centering
    \subfloat[Zybo device]{\includegraphics[width=0.4\textwidth]{images/zybo}\label{fig:device}}
    \hfill
    \subfloat[Zybo resources]{\includegraphics[width=0.4\textwidth]{images/resources}\label{fig:resources}}
    \caption{Resources utilized by LFSR}
    \label{fig:resources}
\end{figure}

\section{Conclusion}
This project has regarded the realization of the Linear-Feedback Shift Register with the feedback polynomial $ 1 + x^{11} + x^{13} + x^{14} + x^{16} $. For the architecture description it has been used the VHDL language. The correctness of the circuit has been tested through simulations using testbenches. It has been done the logic synthesis in order to find the maximum clock frequency and the critical path. Finally, through the implementation run, it has been possible to know which Zybo resources have been used to implement the LFSR.

\newpage

\begin{appendix}
    \listoffigures
    \lstlistoflistings
\end{appendix}

\bibliography{documentation} 
\bibliographystyle{ieeetr}

\end{document}